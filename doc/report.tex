\documentclass{article}
\usepackage[utf8]{inputenc}
\usepackage{graphicx}
\usepackage{amsmath}
\usepackage{amssymb}
\usepackage{amsfonts}
\usepackage{indentfirst}
\usepackage[a4paper, total={7in, 10in}]{geometry}
\graphicspath{ {./../res/} }
\numberwithin{equation}{section}

\title{Nanoparticles size estimation through Mie resonances}
\author{Neven Gentil}
\date{April 2022}

\begin{document}

\maketitle

\twocolumn

\section{Theory}

\subsection{Vector spherical harmonics}

First of all, let's describe the general composition of our problem: given an arbitrary particle droped in a certain medium, we hit it with a monochromatic light. With a fixed size and other specifics parameters for electromagnetic materials like the permeability $\mu$ or the complex refractive index $N$, our objective is to, theoretically, compute the electromagnetic field for the medium surrounding the particle and inside this particle itself, in order to find the expression of the scattered light.

Let's start with an arbitrary electromagnetic field represented by the couple of \textit{monochromatic} vector field $(\textbf{E}, \textbf{H})$ which satisfies the Maxwell equations:
\begin{align}
\nabla \cdot \textbf{E} &= 0\\
\nabla \cdot \textbf{H} &= 0\\
\nabla \times \textbf{E} &= i\omega \mu \textbf{H} \label{eq:rot_e} \\
\nabla \times \textbf{H} &= -i\omega \epsilon \textbf{E} \label{eq:rot_h}
\end{align}
Given the relation $k^{2} = \omega ^{2}\epsilon \mu$ and the following formula:
\begin{align}\label{eq:rot_rot_a}
\nabla \times (\nabla \times \textbf{A}) = \nabla (\nabla \cdot \textbf{A}) - \nabla \cdot (\nabla \textbf{A})
\end{align}
we obtain a couple of \textit{vector wave equation}:
\begin{align}
\nabla ^{2} \textbf{E} + k^{2}\textbf{E}&=0\\
\nabla ^{2} \textbf{H} + k^{2}\textbf{H}&=0
\end{align}
where $\nabla ^{2}$ is the vector Laplace operator, that is to say, we explicitly have $\Delta \textbf{A} = \nabla ^{2} \textbf{A} = (\nabla \cdot \nabla) \textbf{A}$.

An important thing to notice is that any vector field with zero divergence and satisfying the vector wave equation is a valid electric/magnetic field where we can obtain the corresponding magnetic/electric field by the curl (\ref{eq:rot_e})/(\ref{eq:rot_h}).

In this way, it is possible to create a vector function $\textbf{M}$ depending on a scalar function $\psi$ and an arbitrary constant vector $\textbf{c}$:
\begin{align}
\textbf{M} = \nabla \times (\textbf{c}\psi)
\end{align}
where \textbf{M} directly satisfies the condition $\nabla \cdot \textbf{M} = 0$. If we replace the expression of \textbf{M} in the vector wave equation, and thanks to (\ref{eq:rot_rot_a}), we easily obtain:
\begin{align}
\nabla ^{2} \textbf{M} + k^{2}\textbf{M} = \nabla \times [\textbf{c}(\nabla ^{2} \psi + k^{2}\psi)]
\end{align}
To satisfy the vector wave equation, $\psi$ should also satisfy the equivalent scalar wave equation:
\begin{align}\label{eq:psi_wave_eq}
\nabla ^{2} \psi + k^{2}\psi = 0
\end{align}
where, this time, $\nabla ^{2}$ is the scalar Laplace operator. By the way, as described above, if we denote $\nabla \times \textbf{N} = k \textbf{M}$, we have the perpendicular vector field of the associated artificial one, also satisfying the vector wave equation. Finally, we created the vector harmonics $\textbf{M}$ and $\textbf{N}$ with the scalar generating function $\psi$ associated to our initial electromagnetic field.

\subsection{Associated factors}

The main idea to compute all solutions required is the following: given an incident electromagnetic field, we use the boundary conditions on the surface of our particle to get all fields resulting from interaction. Thereby, it could be interesting to use the spherical polar coordinates during the calculation and particularly to express the boundary conditions. That is why we have constructed the vector harmonics above. Indeed, if we replace the constant vector $\textbf{c}$ by the radial coordinate $\textbf{r}$, $\textbf{M}$ is always a solution of the vector wave equation but, this time, for the spherical polar coordinates linked to the particle. In this way, from (\ref{eq:psi_wave_eq}), we obtain:
\begin{equation}\label{eq:wave_sph}
\begin{aligned}
\frac{1}{r^{2}}\frac{\partial }{\partial r}(r^{2}\frac{\partial \psi}{\partial r}) \quad + \\ 
\frac{1}{r^{2}sin(\theta)}\frac{\partial }{\partial \theta}(sin(\theta)\frac{\partial \psi}{\partial \theta}) \quad + \\
\frac{1}{r^{2}sin(\theta)}\frac{\partial^{2} \psi}{\partial^{2} \phi} \quad + \\
k^{2}\psi \quad = 0
\end{aligned}
\end{equation}
Obviously, from the previous formula, we can separate each variable to create $\psi$:
\begin{align}\label{eq:psi_r_t_p}
\psi(r, \theta, \phi) = R(r)\Theta(\theta)\Phi(\phi)
\end{align}
Thereby, when substitued (\ref{eq:psi_r_t_p}) into (\ref{eq:wave_sph}), we obtain three different equations, depending respectively on $r$, $\theta$ and $\phi$ which might be solved separately.

Finally, this kind of generating function $\psi$ satisfying the scalar wave equation (\ref{eq:psi_wave_eq}) is expressed in two \textit{even} ($\psi_{e}$) and \textit{odd} ($\psi_{o}$) functions as follow:
\begin{align}
\psi_{emn}&=cos(m\phi)P_{n}^{m}(cos\theta)z_{n}(kr)\\
\psi_{omn}&=sin(m\phi)P_{n}^{m}(cos\theta)z_{n}(kr)
\end{align}
where the parity of our original function is led by the $cosinus$ and $sinus$ attached with the $\phi$ variable and the \textit{angle-dependent} $m$ variable. 

The other angular part described by $\theta$ , as a \textit{Legendre's differential equation}, is solved by the \textit{associated Legendre functions} of the first kind $P_{n}^{m}(cos\theta)$ orthogonally defined on the $n$ variable. 

Moreover, the radial part, described by $r$, might be reintroduced as follow:
\begin{align}\label{eq:radial_part}
\rho\frac{d }{d\rho}(\rho\frac{d Z}{d\rho})+[\rho^{2}-(n+\frac{1}{2})^{2}]Z=0
\end{align}
where we assume that $\rho=kr$ and $Z=R\sqrt{\rho}$. Thereby, to solve the previous equation, we can use any linear combination of the spherical Bessel functions $j_{n}$, $y_{n}$, $h^{(1)}_{n}=j_{n}+iy_{n}$ or $h^{(2)}_{n}=j_{n}-iy_{n}$: we named this kind of combination $z_{n}$.

Note that $m$ and $n$ are produced by subsidiary conditions when obtaining these three equations by rewriting (\ref{eq:wave_sph}) with separate variables: it is inherent to our physical assumptions.

In the end, we can retrieve our initial vector spherical harmonics generated by $\psi_{emn}$ or $\psi_{omn}$:
\begin{align}
\textbf{M}_{emn}=\nabla \times (\textbf{r}\psi_{emn})\\
\textbf{M}_{omn}=\nabla \times (\textbf{r}\psi_{omn})\\
\textbf{N}_{emn}=\frac{\nabla \times \textbf{M}_{emn}}{k}\\
\textbf{N}_{omn}=\frac{\nabla \times \textbf{M}_{omn}}{k}
\end{align}

At this point of the theory, it is still possible to restrain our hypothesis and more precisely the shape of the incident light. Let's consider a planar incident electromagnetic wave $E_{i}$. Now, we can easily describe this wave in terms of vector spherical harmonics:
\begin{equation}\label{eq:ei_harmo}
\begin{aligned}
\textbf{E}_{i}=\sum_{m=0}^{\infty }\sum_{n=m}^{\infty }B_{emn}\textbf{M}_{emn}+B_{omn}\textbf{M}_{omn}\\
+A_{emn}\textbf{N}_{emn}+A_{omn}\textbf{N}_{omn}
\end{aligned}
\end{equation}
where $B_{X}$ and $A_{X}$ are arbitrary factors. Thanks to the orthogonality of all the vector spherical harmonics which could be demonstrated through properties from $cos(m\phi)$, $sin(m\phi)$ and $P_{n}^{m}(cos\theta)$, we easily exctract each factor of this linear summation (\ref{eq:ei_harmo}):
\begin{align}\label{eq:bemn}
\textbf{B}_{emn}=\frac{\int_{0}^{2\pi}\int_{0}^{\pi}\textbf{E}_{i}\cdot \textbf{M}_{emn}sin\theta d\theta d\phi}{\int_{0}^{2\pi}\int_{0}^{\pi}|\textbf{M}_{emn}|^{2}sin\theta d\theta d\phi}
\end{align}
with here, for instance, the first coefficient. With the orthogonality of the sine and cosine and (\ref{eq:bemn}), we can also demonstrate that $B_{emn}$ and $A_{omn}$ vanishe for all $m$ and $n$. Then, for the same reason, $B_{omn}$ and $A_{emn}$ are nonzero only for $m=1$. Moreover, regarding their behavior and our physical assumptions, we can select the spherical Bessel function replacing $z_{n}$: we choose $j_{n}$ for its finite values close to $r=0$ and denote it as $^{(1)}$ for the associated vector spherical harmonics. Finally, we obtain:
\begin{align}
\textbf{E}_{i}=\sum_{n=1}^{\infty }B_{o1n}\textbf{M}^{(1)}_{o1n} + A_{e1n}\textbf{N}^{(1)}_{e1n}
\end{align}
Afterward, thanks to a bunch of properties from the associated Legendre functions, we deduce explicitly these two remaining factors $B_{o1n}$ and $A_{e1n}$ from (\ref{eq:bemn}) or the equivalent expression:
\begin{align}
\textbf{E}_{i}=E_{0}\sum_{n=1}^{\infty }i^{n}\frac{2n+1}{n(n+1)}(\textbf{M}^{(1)}_{o1n} - i\textbf{N}^{(1)}_{e1n})
\end{align}
Obviously, it is possible to calculate the perpendicular magnetic field $H_{i}$ with the curl (\ref{eq:rot_e}).

\subsection{Boundary conditions}

At this point, we have enough equation relative to the incident plane wave to apply the following boundary conditions:
\begin{align}\label{eq:boundaries}
(\textbf{E}_{i} + \textbf{E}_{s} - \textbf{E}_{I}) \times \overrightarrow{e_{r}} &= 0\\
(\textbf{H}_{i} + \textbf{H}_{s} - \textbf{H}_{I}) \times \overrightarrow{e_{r}} &= 0
\end{align}
where $(\textbf{E}_{s}, \textbf{H}_{s})$ and $(\textbf{E}_{I}, \textbf{H}_{I})$ are respectively the scattering and internal electromagnetic field resulting from the interaction between the incident field and our particle. Then, we assume that these last fields have the same mathematical shape, when developped in vector spherical harmonics, than $(\textbf{E}_{i}, \textbf{H}_{i})$; that is to say, the scattering and internal field are linearly dependent from the incident field up to some factors taking into account the behavior of $j_{n}$ at the boundary of our particle:
\begin{align}
\textbf{E}_{I}&=\sum_{n=1}^{\infty }E_{n}(c_{n}\textbf{M}^{(1)}_{o1n} - id_{n}\textbf{N}^{(1)}_{e1n})\\
\textbf{H}_{I}&=\frac{-k_{I}}{\omega\mu_{I}}\sum_{n=1}^{\infty }E_{n}(d_{n}\textbf{M}^{(1)}_{e1n} + ic_{n}\textbf{N}^{(1)}_{o1n})\\
\textbf{E}_{s}&=\sum_{n=1}^{\infty }E_{n}(ia_{n}\textbf{N}^{(3)}_{e1n} - b_{n}\textbf{M}^{(3)}_{o1n})\\
\textbf{H}_{s}&=\frac{k}{\omega\mu}\sum_{n=1}^{\infty }E_{n}(ib_{n}\textbf{N}^{(3)}_{o1n} + a_{n}\textbf{M}^{(3)}_{e1n})
\end{align}
Note that $^{(3)}$ means we use the spherical Bessel function of the third kind $h^{(1)}_{n}$ and $\textbf{H}_{X}$ has the same factors than $\textbf{E}_{X}$ due to the curl (\ref{eq:rot_e}). Moreover, $k_{I}$ and $\mu_{I}$ are respectively the wave vector and the permeability of the particle.

Now, we can rewrite the boundary conditions on the surface of our imaginary particle from (\ref{eq:boundaries}) with each component separated, that is to say at $r=a$ if we denote by $a$ its radius:
\begin{equation}\label{eq:boundary_comp}
\begin{aligned}
E_{i\theta} + E_{s\theta} = E_{I\theta}\\
E_{i\phi} + E_{s\phi} = E_{I\phi}\\
H_{i\theta} + H_{s\theta} = H_{I\theta}\\
H_{i\phi} + H_{s\phi} = H_{I\phi}
\end{aligned}
\end{equation}
where $E_{iX}$, for instance, denotes the $X$-component of $\textbf{E}_{i}$. Then, we apply the explicit expressions for the incident, scattering and internal field on these previous boundary conditions (\ref{eq:boundary_comp}) with, needless to say, the detailled expressions of the vector spherical harmonics $\textbf{M}_{X}$ and $\textbf{N}_{X}$. In this way, we obtain four linear equations revealing the four coefficients arbitrary pushed into $(\textbf{E}_{I}, \textbf{H}_{I})$ and $(\textbf{E}_{s}, \textbf{H}_{s})$:
\begin{equation}
\begin{aligned}
j_{n}(mx)c_{n} + h^{(1))}_{n}(x)b_{n} = j_{n}(x) \\
\mu[mxj_{n}(mx)]'c_{n} + \mu_{I}[xh_{n}(x)]'b_{n} = \mu_{I}[xj_{n}(x)]' \\
\mu mj_{n}(mx)d_{n} + \mu_{I}h^{(1))}_{n}(x)a_{n} = \mu_{I}j_{n}(x) \\
[mxj_{n}(mx)]'d_{n}+m[xh^{(1))}_{n}(x)]'a_{n} = m[xj_{n}(x)]'
\end{aligned}
\end{equation}
where the prime sign means derivation along the same argument between parenthesis. Moreover, we have the following definitions:
\begin{equation}\label{eq:def_x_m}
\begin{aligned}
x = ka = \frac{2\pi Na}{\lambda} \qquad m = \frac{N_{I}}{N}
\end{aligned}
\end{equation}
where $N_{I}$ indicates the refractive index of the particle. In the same way, $N$ is owned by the medium. By the way, it could be intersting to notice that we also have $\rho(a)=x$. 

Then, thanks to a basic Gaussian elimination, we can exctract each coefficient. Surprisingly, with some approximation on the spherical Bessel functions depending on the frequency, it is possible to admit the equality between $a_{n}$ and $d_{n}$ then $b_{n}$ and $c_{n}$. Furthermore, these two factors could be simplified by introducing the \textit{Riccati-Bessel} functions:
\begin{align}
\psi_{n}(\rho)=\rho j_{n}(\rho) \qquad \xi_{n}(\rho)=\rho h^{(1)}_{n}(\rho)
\end{align}
and we finally obtain the following expressions for the two first coefficients:
\begin{equation}\label{eq:an_bn}
\begin{aligned}
a_{n} = \frac{m\psi_{n}(mx)\psi^{'}_{n}(x)-\psi_{n}(x)\psi^{'}_{n}(mx)}{m\psi_{n}(mx)\xi^{'}_{n}(x)-\xi_{n}(x)\psi^{'}_{n}(mx)}\\
b_{n} = \frac{\psi_{n}(mx)\psi^{'}_{n}(x)-m\psi_{n}(x)\psi^{'}_{n}(mx)}{\psi_{n}(mx)\xi^{'}_{n}(x)-m\xi_{n}(x)\psi^{'}_{n}(mx)}
\end{aligned}
\end{equation}

\subsection{Cross sections}

From now on, we have enough theory to think about the experimental approach and fit all our requirements. That is to say, let's describe the coordinate system through an explicit diagram:
\begin{figure}[h]
    \centering
    \includegraphics[width=0.4\textwidth, height=0.4\textwidth]{system.png}
    \caption{Coordinate System}
    \label{fig:system}
\end{figure}
As we can see, this reference frame is really common and our incomming plane wave is going through our particle along the z-axis. That said, our actual way to mesure the particle's size is to, roughly, retrieve the power emitted by it thanks to an optical microscope. Thereby, our goal is currently is to calculate this associated power, that is to say the net rate at which electromagnetic field crosses the surface A of an arbitrary sphere, which might be expressed as follow:
\begin{align}\label{eq:power_def}
W=\int_{A}^{}\textbf{S}\cdot \overrightarrow{e_{r}}dA
\end{align}
where $\textbf{S}$ is the \textit{Pointing vector} defined as:
\begin{align}
\textbf{S} = \frac{1}{2}Re\left\{\textbf{E}^{*} \times \textbf{H}\right\}
\end{align}
Note that an arbitrary sign can be added to (\ref{eq:power_def}) in fonction of our physical assumptions. In this way, we can now compute this power for the scattered electromagnetic field $(\textbf{E}_{s}, \textbf{B}_{s})$ and we obtain:
\begin{align}\label{eq:w_s}
W_{s}=\frac{1}{2}Re\left\{ \int_{\phi=0}^{2\pi}\int_{\theta=0}^{\pi} (E_{s\theta}H^{*}_{s\phi} - E_{s\phi}H^{*}_{s\theta})r^{2}sin\theta d\theta d\phi\right\}
\end{align}
which might be easily calculated thanks to (\ref{eq:an_bn}) included into the expression of the scattered field and, after a bit of mathematical manipulations, we have:
\begin{align}
W_{s}=\frac{\pi\left| E_{0} \right|^{2}}{k\omega\mu}\sum_{n=1}^{\infty }(2n+1)Re\left\{ g_{n} \right\}(\left| a_{n} \right|^{2}+\left| b_{n} \right|^{2})
\end{align}
where $g_{n}=(\chi^{*}_{n}\psi^{'}_{n}-\psi^{*}_{n}\chi^{'}_{n})-i(\chi^{*}_{n}\chi^{'}_{n}+\psi^{*}_{n}\psi^{'}_{n})$. This last definition can be simplified by the identity $\chi_{n}\psi^{'}_{n}-\psi_{n}\chi^{'}_{n}=1$. Moreover this power could be \textit{normalized}, that is to say, divided by the intensity from the incident light. Thereby, we introduce the associated cross section:
\begin{align}\label{eq:csca}
C_{sca}=\frac{W_{s}}{I_{i}}=\frac{2\pi}{k^{2}}\sum_{n=1}^{\infty }(2n+1)(\left| a_{n} \right|^{2}+\left| b_{n} \right|^{2})
\end{align}

In the same way, all of these previous step from (\ref{eq:w_s}) might be re-used for any other fields instead of the scattered field. Until now, only the light emitted by the particle was studied thanks to three fields: $(\textbf{E}_{i}, \textbf{H}_{i})$, $(\textbf{E}_{s}, \textbf{H}_{s})$ and $(\textbf{E}_{I}, \textbf{H}_{I})$. However, it is possible to express a variable of extinction, opposed to the scattering \textit{power} $W_{s}$, due to the interaction between the electromagnetic plane wave and our particle:
\begin{align}\label{eq:wext_ws_wa}
W_{ext} = W_{s} + W_{a}
\end{align}
This additional term, combined with the new one $W_{a}$, should be easily understood through a \textit{power balance}, that is to say an energy balance replaced by the equivalent power. Indeed, if we denote $W_{a}$ the amount of power which was varying during the interaction ($W_{a} > 0$ means that power is absorbed within the particle whereas $W_{a} > 0$ means that power is being created within the particle), it might be possible to write the following:
\begin{align}
W_{a} = W_{i} - W_{s} + W_{ext}
\end{align}
As our actual medium is nonabsorbing, $W_{i}$ vanishes identically and we retrieve the relation (\ref{eq:wext_ws_wa}). Now, the extinction \textit{power} might be evaluated in fonction of our previous vector spherical harmonic terms:
\begin{equation}
\begin{aligned}
W_{s}=\frac{1}{2}\int_{\phi=0}^{2\pi}\int_{\theta=0}^{\pi} (&Re\{E_{i\phi}H^{*}_{s\theta} - E_{i\theta}H^{*}_{s\phi}\}\; +\\
&Re\{E_{s\phi}H^{*}_{i\theta}) - E_{s\theta}H^{*}_{i\phi}\}) \\
&r^{2}sin\theta d\theta d\phi
\end{aligned}
\end{equation}
and, as stated above in (\ref{eq:csca}), we finally obtain the extinction cross section by proportionality:
\begin{align}\label{eq:cext}
C_{ext}=\frac{W_{ext}}{I_{i}}=\frac{2\pi}{k^{2}}\sum_{n=1}^{\infty }(2n+1)Re\left\{a_{n} + b_{n} \right\}
\end{align}

At last, we have calculated and theoretically established the expression of two general terms which we might be retrieved by experiments: $C_{sca}$ and $C_{ext}$. It is important to note that these cross section are, not only dependent on, but also strongly linked to our hypothesis, that is to say our input variables like the refractive index. Thereby, $C_{sca}$ and $C_{ext}$ are directly reliant on the size of our studied particle.

\section{Simulations}

\subsection{Equations adjustment}

First at all, our main issue to numerically compute the cross sections $C_{sca}$ and $C_{ext}$ is to calculate the associated scattering coefficients $a_{n}$ and $b_{n}$. Globally, these two numbers are obtained through Bessel functions as we can see from (\ref{eq:an_bn}) and our goal is to retrieve each value of them between a certain range of wavelength in order to obtain a kind of spectrum. However, due to the size of our nanoparticle close to the incident wavelength included in a visible range, the variable $\rho=kr$, as a dependency for the Bessel functions, is also restrained into a small range of value close to zero along the x-axis. In fact, in the context of a computer, these targeted values could be too sensitive to be calculated with a reasonably high precision. Thereby, we could introduce the \textit{logarithmic derivative}:
\begin{align}
D_{n}(\rho)=\frac{d}{d\rho}ln[\psi_{n}(\rho)]
\end{align}
Thanks to this additional definition, we are expanding the range of hit values and decreasing the sensitivity. In this way, we can rearrange the expressions of $a_{n}$ and $b_{n}$ from (\ref{eq:an_bn}):
\begin{equation}
\begin{aligned}
a_{n}&=\frac{[D_{n}(mx)/m + n/x]\psi_{n}(x)-\psi_{n-1}(x)}{[D_{n}(mx)/m + n/x]\xi_{n}(x)-\xi_{n-1}(x)}\\
b_{n}&=\frac{[mD_{n}(mx) + n/x]\psi_{n}(x)-\psi_{n-1}(x)}{[mD_{n}(mx) + n/x]\xi_{n}(x)-\xi_{n-1}(x)}
\end{aligned}
\end{equation}
where we have used the recurrence relation:
\begin{align}
z^{'}_{n}(x)=z_{n-1}(x)-\frac{nz_{n}(x)}{x}
\end{align}
with $z_{n}$ replaced by $\psi_{n}$ or $\xi_{n}$. Note that $x$ is not a constant because of its dependency through $\lambda$ that we are going to slide along our targeted range. Moreover, due to the recurrence relations of Bessel functions explained below, in the next subsection, $D_{n}$ also satisfies the following relation:
\begin{align}\label{eq:rec_dn}
D_{n-1}(\rho)=\frac{n}{\rho}-\frac{1}{D_{n}(\rho)+n/\rho}
\end{align}
Because of the link existing between $D_{n}$ and Bessel functions, this previous recurrence (\ref{eq:rec_dn}) should be runned downwardly, that is to say begining with the higher $n$ owned by the scattering coefficients, in order to avoid the unstability of our computation as stated in the subsection below. Otherwise, the calculated data could be rapidly \textit{flawed} while $n$ is increasing.

\subsection{Computation of Bessel functions}

Then, our last issue is to numerically compute the spherical Bessel functions, as the first, second and the associated Hankel functions or even the Ricatti-Bessel functions. Hopefully, in our preferred scientific language that is \textit{Python}, all of these features are already implemented. Thereby, as a first approach and taking into account that \textit{Mie Theory} is not really heavy, it is completly useless and counterproductive to re-write this mathematic area. However, it could be interesting to speed up the computation thanks to a low-level programming language, or more generally, a compiled one. At least, for a better understanding of the computation, we can define a way to numerically implement these tools.

To contextually replace our problem, these Bessel functions are introduced in our theory through the resolution of the radial part (\ref{eq:radial_part}) in order to get a global expression of the generating couple of function $\psi_{emn}$ and $\psi_{omn}$. Involving spherical coordinates in our calculation, we must use the spherical Bessel functions $j_{n}$ and $y_{n}$ which are a special case of general Bessel functions $J_{n}$ and $Y_{n}$ solving the Bessel's differential equation:
\begin{align}
x^{2}\frac{d^{2}y}{dx^{2}} + x\frac{dy}{dx} + (x^{2} - \alpha^{2})y = 0
\end{align}
where $\alpha$ is a complex number without restriction. Thereby, we have the following definitions:
\begin{align}
j_{n}(x)=\sqrt{\frac{\pi}{2x}}J_{n+\frac{1}{2}}(x) \qquad y_{n}(x)=\sqrt{\frac{\pi}{2x}}Y_{n+\frac{1}{2}}(x)
\end{align}
Note that the actual $x$ might be real or a complex value and is, by definition, different from all previous sections.

After some mathematics calculation, it is possible to demonstrate the following recurrence relations:
\begin{align}\label{eq:rec_sph_bess}
\frac{2n+1}{x}z_{n}(x) &= z_{n-1}(x) + z_{n+1}(x)\\
(2n+1)\frac{d}{dx}z_{n}(x) &= nz_{n-1}(x) - (n+1)z_{n+1}(x)
\end{align}
where $z_{n}$ is either $j_{n}$ or $y_{n}$. Thanks to these previous equations (\ref{eq:rec_sph_bess}), and given the two first exact expression for the spherical Bessel functions, we can theoretically obtain all wished order $n$. So, \textit{by hand}, we have the following:
\begin{equation}
\begin{aligned}
j_{0}(x)=\frac{sin(x)}{x} \qquad &j_{1}(x)=\frac{sin(x)}{x^{2}} - \frac{cos(x)}{x}\\
y_{0}(x)=-\frac{cos(x)}{x} \qquad &y_{1}(x)=-\frac{cos(x)}{x^{2}}-\frac{sin(x)}{x}
\end{aligned}
\end{equation}
with the associated figures:
\begin{figure}[h]
    \centering
    \includegraphics[width=0.5\textwidth, height=0.5\textwidth]{jnyn.png}
    \caption{$j_{0}$, $j_{1}$ with $y_{0}$, $y_{1}$}
    \label{fig:jnyn}
\end{figure}

Thereby, by inverting the main recurrence relations (\ref{eq:rec_sph_bess}) into this way:
\begin{align}\label{eq:sph_bess_up_rec}
z_{n+1}(x) &= \frac{2n+1}{x}z_{n}(x) - z_{n-1}(x)
\end{align}
we should be able to compute the next order $n=2$ for a given $x$ either real or complex. Continuing with the same idea, we can obtain $j_{3}$, $j_{4}$... coupled with the associated $y_{n}$. This way to compute all wished spherical Bessel function, or more generally any function recursively defined, from the lowest order $n=0$ reaching the highest order $n=\infty$, is commonly called an \textit{upward recurrence}. Assuming that we have in our possession a couple of spherical Bessel function at the order $n=i$ and $n=i+1$, an opposed computation from the previous one, should allow us to recursively reach $n=0$: 
\begin{align}\label{eq:sph_bess_down_rec}
z_{n-1}(x) &= \frac{2n+1}{x}z_{n}(x) - z_{n+1}(x)
\end{align}
that is called a \textit{downward recurrence} because of the decrement of $n$ for each calculated value. Despite these two recursive ways are strictly equivalent from a mathematical point of view, there is a not negligible difference between them. Indeed, taking into account boundaries of coding number (currently 64 bit for a common \textit{float}) and the behavior of spherical bessel functions, it is possible, but tough, to demonstrate either the stability or the unstability for the first and second kind calculated through downward or upward reccurency. Otherwise, by a \textit{hand} way, we can easily retrieve these theoritical results by comparing our calculations with exact curves already tabulated:
\begin{itemize}
\item Stable by \textbf{upward} recurrency: 
\begin{itemize}
\item[*]$y_{n}, \forall x$
\item[*]$j_{n}$, if $|x| \geqslant n$
\end{itemize}
\item Stable by \textbf{downward} reccurency:
\begin{itemize}
\item[*]$j_{n}$, if $|x| < n$
\end{itemize}
\end{itemize}
assuming that $Re\{x\} \in\mathbb{R}_{+}$. Now, we can think about the programming steps in order to compute our spherical Bessel functions of the first and second kind through an upward recurrency: we get $j_{0}$/$y_{0}$ and $j_{1}$/$y_{1}$ from a given $x$ then, thanks to (\ref{eq:sph_bess_up_rec}), we compute all required order $j_{n}$/$y_{n}$ with $n \geqslant 2$. However, a question remains in a downward reccurency: How to get starting Bessel function values for higher order $n=i$ and $n=i+1$?

Theoretically, it is well know that Bessel functions of the first kind $J_{n}$ satistfies the following relation:
\begin{align}
\forall\,x\,|\,Re\{x\} \in\mathbb{R}_{+}, \lim_{n \to \infty } J_{n}(x)=0
\end{align}
Thereby a common approach, called the \textit{Miller} method, is to suppose that our targeted higher order $i$ is arbitrary equal to $1$ whereas the $i+1$ order is null. Afterward, we can easily pull down the order thanks to (\ref{eq:sph_bess_down_rec}). Once we obtain the first order $n=0$, because of the linearity of our recurrence relations, we \textit{normalize} this serie by the true value of $j_{0}(x)$. 

For example:
\begin{equation}\label{eq:j_computed}
\begin{aligned}
j^{computed}_{i+1}(x) &= 0\\
j^{computed}_{i}(x) &= 1\\
j^{computed}_{i-1}(x) &= \frac{2i+1}{x}j^{computed}_{i}(x) - j^{computed}_{i+1}\\
&...\\
j^{computed}_{0}(x) &= \alpha
\end{aligned}
\end{equation}
Now, we can \textit{normalize}:
\begin{equation}\label{eq:j_final}
\begin{aligned}
j_{0}(x) &= \frac{sin(x)}{x} \qquad\textnormal{(by definition)}\\
j^{final}_{0}(x) &= j^{computed}_{0}(x) \times \frac{j_{0}(x)}{\alpha} = j_{0}(x)\\
j^{final}_{1}(x) &= j^{computed}_{1}(x) \times \frac{j_{0}(x)}{\alpha}\\
&...\\
j^{final}_{i}(x) &= j^{computed}_{i}(x) \times \frac{j_{0}(x)}{\alpha}\\
\end{aligned}
\end{equation}
where $j^{final}_{X}$ is the wished result. In fact, we just have \textit{normalized} our serie $j^{computed}_{X}$ by the initial factor $\frac{j_{0}(x)}{\alpha}$. It is important to notice that any serie constructed through this way always verifies the recurrence relation (\ref{eq:sph_bess_up_rec}) and each term is, by definition, a solution of the radial part (\ref{eq:radial_part}).

Unfortunately, a last issue still remain in our downward recurrency approach. Something that is not obvious at the first glance, we see that for a given $i$ and for any $x$, the serie $j^{computed}_{X}$ always has the same value for all terms in it. That is to say, only the \textit{normalization} factor introduce the value $x$ in our final serie and, finally, $j^{computed}_{X}(x)$ is not really dependent on $x$ (we can omit this parameter afterwards). In this way, due to the mathematical shape of our recurrence relation (\ref{eq:sph_bess_down_rec}), each value of a lower computed order is bigger than its previous higher order: $j^{computed}_{0} > j^{computed}_{1} > ... > j^{computed}_{i}$. So, if we target a really high order with, for instance, $i=10^{10}$, the first order will ineluctably get a really high value. Thereby, for a computer with \textit{physical} boundaries for its numbers, this previous result will give us an invalid value for an intermediate order and all order below will be saturated with the same invalid state. In this way, the \textit{normalization} step is impossible because $\alpha$ in (\ref{eq:j_final}) will own a \textit{NaN} value (that is a computer overflow indicator). Even if, for a first approximation, we use the ten first order for calculation, this phenomenon will be rapidly visible even for our range of value (from $n=50$) and it could be intersting to avoid this \textit{blocking} state. So, the first solution proposed in many other articles treating the \textit{Mie Theorie}, or more generally the computation of Bessel functions, is to set an arbitray minimal value for $i$, that is to say $10^{-10}$ instead of $1$. However, this method will only postpone the phenomenon. A second solution is to divide the downward reccurency into multiple chunks. As mentioned above, $j^{computed}_{X}$ is recursive and completly dependent on the targeted order $i$. In fact, if we want $i=1000$ as a final order, we can calculate $j^{computed}_{X}$ from $n=0$ to $n=50$ (stability limit) and apply a normalization as previously stated for this first chunk. Secondly, we compute a second chunk starting from $n=50$ to $n=100$ (actually, the computation begins with $n=100$ because of the downward behavior). Then, it is possible to assume that $\{j^{computed}_{0}...j^{computed}_{50}\} = \{j^{computed}_{50}...j^{computed}_{100}\}$ due to the arbitrary recurrence relation. For the second chunk, we can normalize it by $j^{computed}_{50}$ from the first chunk, itself normalized by the exact $j_{0}(x)$ (which is not arbitrary but dependent on $x$). This method is chain-shaped and we recursively advance, chunk by chunk to finally obtain every order wished without overflowing the computer results.

\subsection{Input variables}

At this point of our simulation part, we have to precisely fix and set our input variables, that is to say with, roughly, this following pattern:
\begin{align}
Output = Simulation(Input)
\end{align}
where $Output$ is our wished scattering and extinction cross sections. Given the relations (\ref{eq:csca}) and (\ref{eq:cext}) coupled with the expressions of $a_{n}$ and $b_{n}$ previously established, we are able to recognize these variables. Thereby, all inputs are included into (\ref{eq:def_x_m}) with refractive indices of the medium $N_{I}$ and the particle $N$, the particle's size $a$ then the wavelength $\lambda$.

First at all, let's consider the $\lambda$ variable. The problem is the following: from our theory, we send an incident monochromatic light, that is to say $\lambda=constant$ in the visible range. However, in reality, the experimental tool produce multiple monochromatic lights in order to reach all values in the visible range and compute the associated cross sections. In fact, the wavelength $\lambda$ will vary for each computation and, by the way, is an $Input$ variable. Therefore, we have $x=x(\lambda)$ and more precisely $k=k(\lambda)$

Secondly, the particle's size $a$ represents the radius of this last one. Obviously, from the purpose of this study, it is also an $Input$ variable.

Then, the refractive index of the medium, taking into account that the environment is homogenous, this last one is a constant value. Even more, if we consider that the medium is the ambiant air, it might be possible to approximate it as the void and finally set $N_{I}=1$. Moreover, we have now the following relation $m=N$.

The last variable to describe is the refractive index $N$ of our particle. At this point, we can introduce the targeted material, that is to say, the substance used into our spherical sample: silicon. As a first approximation, we can say that silicon has a constant refractive index commonly setted around $3.5$. Thereby, for $a=100^{-9}m$ and $\lambda=[206^{-9}, 826^{-9}]$ (visible range a bit extended in the \textit{UV} range), we obtain the following cross sections:
\begin{figure}[h]
    \centering
    \includegraphics[width=0.5\textwidth, height=0.4\textwidth]{ri_const.png}
    \caption{Cross sections with $N=3.5$}
\end{figure}
where $Sca$ is the scattering, $Ext$ the extinction and $Abs$ the absorbtion cross section linked with (\ref{eq:wext_ws_wa}): $C_{abs} = C_{ext} - C_{sca}$. We can see that there is no absorbtion and the scattering cross section equals the extinction one. However, to properly describe physical states of matter, we have to consider that the refractive index is varying depending on the wavelength of the incident light, that is to say, $N=N(\lambda)$. In this way we obtain:
\begin{figure}[h]
    \centering
    \includegraphics[width=0.5\textwidth, height=0.4\textwidth]{ri_var_real.png}
    \caption{Cross sections with $N=N(\lambda)$}
\end{figure}
where we use the real part of pre-computed refractive indices from another scientific experiment (Aspnes and Studna, 1983). We easily see that the main maximums, called \textit{Mie resonances}, are shifted to the right but this approach globally keep the previous shape. It is always remarkable that $C_{sca}=C_{ext}$. That is due to the real expression of the refractive index. Indeed, the absorbtion behavior of our material is mathematicaly implemented through the imaginary part of $N$. In this way, we should have $N=N_{re}+iN_{im}$ Thereby, when we use a complex varying index, we obtain:
\begin{figure}[h]
    \centering
    \includegraphics[width=0.5\textwidth, height=0.4\textwidth]{ri_var_complex.png}
    \caption{Cross sections with $N=N(\lambda)$ complex}
\end{figure}
Now, in addition to the previous shape then according to each Mie's resonance pike, we have a maximum of absorbtion describing the capacity of our particle to redistribute the energy of the incident light into another form, like a thermal energy. It is important to notice that the varying imaginary part doesn't change the position of the pikes in contrary to the varying real part.

Finally, our $Input$ variables are the following: $a$ (particle's size), $\lambda$ (incident light's wavelength) and $N(\lambda)$ (refractive index depending on $\lambda$).

\section{Experiments}

\section{Conclusion}

\end{document}